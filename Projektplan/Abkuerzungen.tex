% !TEX root = Projektdokumentation.tex

% Es werden nur die Abkürzungen aufgelistet, die mit \ac definiert und auch benutzt wurden. 
%
% \acro{VERSIS}{Versicherungsinformationssystem\acroextra{ (Bestandsführungssystem)}}
% Ergibt in der Liste: VERSIS Versicherungsinformationssystem (Bestandsführungssystem)
% Im Text aber: \ac{VERSIS} -> Versicherungsinformationssystem (VERSIS)

% Hinweis: allgemein bekannte Abkürzungen wie z.B. bzw. u.a. müssen nicht ins Abkürzungsverzeichnis aufgenommen werden
% Hinweis: allgemein bekannte IT-Begriffe wie Datenbank oder Programmiersprache müssen nicht erläutert werden,
%          aber ggfs. Fachbegriffe aus der Domäne des Prüflings (z.B. Versicherung)

% Die Option (in den eckigen Klammern) enthält das längste Label oder
% einen Platzhalter der die Breite der linken Spalte bestimmt.
\begin{acronym}[WWWWW]
	\acro{AT Mode}{Transparent Mode}
	\acro{API Mode}{Application Programming Interface Mode}
	\acro{Rx-Schnittstell}{Receiver-Schnittstelle}
	\acro{Tx-Schnittstell}{Transmitter-Schnittstelle}
	\acro{GPIO-Pins}{General-Purpose-Input/output-Pins}
	\acro{ISP-Schnittstelle}{In-System-Programmierung-Schnittstelle}
	\acro{PID-Regler}{Proportional–Integral–Derivative Controller}
	\acro{AMQP}{Advanced Message Queuing Protocol}
\end{acronym}
