\section{Ausgangssituation}
Es handelt sich bei dem Projekt Hausautomatisierung mit ZigBee um ein Thema, dass über mehrere Semester von unterschiedlichen Studentengruppen weitergeführt wird. Bei der Übernahme im Sommersemester 2016 konnte an einem Stand vom Sommersemester 2015 angeknüpft werden. Im folgenden Kapitel wird dieser Stand analysiert.
\begin{figure}[htb]
\centering
\includegraphicsKeepAspectRatio{Ausgang.png}{0.8}
\caption{Projektaufbau der Vorgruppe aus dem Sommersemester 2015}
\label{fig:Ausgang}
\end{figure}

\subsection{Konzepte}\label{Konzepte}
In diesem Abschnitt werden die Konzepte betrachtet und beurteilt, die bei der Vorgruppe erdacht und umgesetzt wurden. Positive sowie negative Aspekte werden dabei kurz erläutert.

\paragraph{REST API}\mbox{}\\
Sämtliche Kommandos an die Endgeräte können über eine RESTful API getätigt werden. Der Einsatz einer API fördert die Erweiterbarkeit des Systems. So können beispielsweise unterschiedliche Frontends entwickelt werden ohne Geschäftslogik redundant zu programmieren.

\paragraph{ZigBee-Kommunikation im AT-Modus}\label{AusgZigBee}\mbox{}\\
Die Kommunikation im Netzwerk findet im \ac{AT Mode} statt. Dabei werden empfangene Daten von der \ac{Rx-Schnittstell} direkt an die \ac{Tx-Schnittstell} weitergeleitet. Damit eine Kommunikation mit mehreren Teilnehmern möglich ist werden alle Nachrichten an die Broadcast-Adresse gesendet. Dadurch ist es notwendig, dass die Adressierung im Protokoll des Hausautomatisierungssystems abgehandelt werden muss. Darüberhinaus sind keine Empfangsbestätigungen im \ac{AT Mode} implementiert. 

Alternativ zum \ac{AT Mode} kann der \ac{API Mode} verwendet werden. Dieser bringt folgende Vorteile\footnote{http://knowledge.digi.com/articles/Knowledge\_Base\_Article/What-is-API-Application-Programming-Interface-Mode-and-how-does-it-work}
\begin{itemize}
\item Ändern von Parametern ohne den Command Mode zu aktivieren
\item Anzeigen von RSSI und Absenderadresse
\item Empfangen von Übertragungsbestätigungen
\end{itemize}

\paragraph{Generisches Protokoll zwischen Server und Endgeräten}\label{AusgProt}\mbox{}\\
Es wurde ein simples Protokoll erarbeitet, das es möglich macht, dass unterschiedliche Geräte gesteuert werden. Das Protokoll kommt dabei mit drei Befehlen aus:
\begin{itemize}
\item Statusanfrage
\item Statusantwort
\item Einen Wert setzen
\end{itemize}

Die Werte, die gesetzt werden können, haben dabei abhängig vom jeweiligen Gerät unterschiedliche Bedeutungen. So ist eine 1 bei einem Lichtschalter An. Bei einem Heizungsventil würde dies eine Öffnung um 1\% bewirken.

Das Protokoll ist jedoch in der Eigenschaft limitiert, dass sich Geräte nicht selbstständig beim Server registrieren können. Eine händische Eingabe des Benutzers ist erforderlich, der über die API den Typ des Geräts festlegt. Hier wäre eine Erweiterung des Protokolls sinnvoll, so dass die Geräte dem Server ihren Typ mitteilen können.

\paragraph{Prototypische Entwicklung mit Arduino}\mbox{}\\
Sowohl Heizungssteuerung als auch Temperatursensor wurden jeweils auf einem Arduino Uno entwickelt. Aufgrund der veringerten Komplexität der Boards können rasch Ergebnisse erzielt werden. Dies ist bei der Evaluierung von Konzepten hilfreich, bevor sie auf dem Zielgerät entwickelt werden.

\subsection{Eingesetzte Hardware}
Eine genau Liste der eingesetzen Hardware kann der Dokumentation der Vorgruppe entnommen werden. Aufgrund des prototypischen Charakters des Projekts bis zu diesem Zeitpunkt war die Entwicklung auf Basis des Arduino Unos sinnvoll. Nach Beendigung der Arbeit der Gruppe in diesem Semester soll ein Stand erreicht sein, der dem eines einsetzbaren Produkts nahe kommt. Dies erfordert den Einsatz kompakter Hardware, die im Fall des Heizungsventils zum Beispiel in einem Gehäuse untergebracht werden kann.
Ein Aufbau des Prototypensystems vom Sommersemester 2015 kann in \Abbildung{Ausgang} betrachtet werden.

\subsection{Sinnvolle Ergänzungen}
Neben der Beurteilung der Konzepte in \autoref{Konzepte} werden in diesem Abschnitt sinnvole Ergänzungen dazu erläutert. Zusammen mit den Gedanken aus \autoref{Konzepte} sollen diese in die Weiterentwicklung einfließen.
\paragraph{Entwicklung eines Frontends}\mbox{}\\
Die Steuerung der Geräte über direkte API-Aufrufe ist für Benutzer des Hausautomatisierungssystems nicht konfortabel. Die Entwicklung eines Frontends ist sinnvoll. Dabei sollte die Einsatzmöglichkeit auf verschiedenen Geräten bedacht werden.
\paragraph{Implementierung einer Authentifizierungs- und Autorisierungsmöglichkeit}\mbox{}\\
Das System im aktuellen Stand ist nicht gegen unberechtigten Zugriff abgesichert. Eine Authentifizierungs- und Autorisierungsmöglichkeit ist für den Einsatz unter reellen Bedingungen zwingend erforderlich.
\paragraph{Zugriff auf Geräte über das Internet}\mbox{}\\
Der Server in der jetzigen Ausführung ist nur über das LAN erreichbar. Für eine Erreichbarkeit über das Internet wäre es notwendig, dass der Benutzer über eine feste IP-Adress verfügt. Besser wäre hier die Schaffung eines Cloud Backends, das mit dem Heimserver kommunizieren kann. Dies ermöglicht auch die Verwaltung mehrerer Gebäude über eine zentrale Plattform.


