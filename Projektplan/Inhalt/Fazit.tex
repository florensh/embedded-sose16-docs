\section{Fazit}
Durch die Bearbeitung des Themas Hausautomatisierung mit ZigBee der Studengruppe im Sommersemester 2016 konnte ein ein Projektstand erreicht werden, der dem Ziel eine einseztbare Plattform zu schaffen ein großes Stück näher gekommen ist. Es wurde eine REST API geschaffen, die es ermöglicht Räume und Geräte zu verwalten. Auf dem Raspberry Pi wurde eine Software installiert, die bidirektional mit dem Backend und den Endgeräten kommunizieren kann. Die ZigBee-Kommunikation findet im \ac{API Mode} statt. Dadurch können unterschiedliche Endgeräte direkt adressiert werden ohne im Broadcast-Modus zu arbeiten. Desweiteren existiert ein optisch ansprechendes Frontend, dass diversen Anzeigegeräten automatisch anpasst. Für die Kommunikation innerhalb des gesamten Systems wurden moderne Technologien verwendet wie zum Beispiel \ac{AMQP} oder Web Sockets. 
Das System ist über das Internet erreichbar und ist durch einen Authentifizierungs- und Autorisierungsmechanismus abgesichert. Für das Ventil wurde ein zweckmäßiges Gehäuse gefertigt, dass die Hardware in sich kapselt. Das Ventil kann prozentual geöffnet werden. Dies ermöglicht den Einsatz eines \ac{PID-Regler} zum Einstellen eines konstanten Raumklimas.

Der Algorithmus für den \ac{PID-Regler} kann von einer Nachfolgegruppe einfach in die Raumserversoftware integriert werden. Auf der Hausautomatisierungsplattform können ebenfalls neue Hardwareprojekte aufbauen. So ist es Möglich, dass zusätzliche Endgeräte entwickelt werden, die das hier konzipierte Protokoll unterstützen.
