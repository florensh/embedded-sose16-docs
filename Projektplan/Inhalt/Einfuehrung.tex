\section{Einführung}
Im Zuge der Vorlesung \betreff\ des Studiengangs \ausbildungsberuf\ wird ein Projekt im Bereich der Hausautomatisierung durchgeführt.
Thema des Projekts ist die Konzeption und Erstellung einer Heizungssteuerung. Dabei wird an einem im Sommersemester 2015 durchgeführten Projekt angeknüpft.

Die Use-Cases des Hausautomatisierungssystems sind allgemein betrachtet, das Lesen und Setzen von Werten für Geräte des häuslichen Gebrauchs. Darunter fallen zum Beispiel Lichtschalter, Türöffner oder wie in unserem konkreten Fall eine Heizung.
Die Schwierigkeit ein solches System zu konzipieren liegt darin mit vielen verschiedenartigen Endgeräten umgehen zu können und gleichzeitig eine intuitiv bedienbare Software zu liefern. Im Idealfall muss der Benutzer das Gerät, dass er in die Hausautomatisierungsumgebung integrieren möchte nur einschalten und das Gerät macht sich in der Umgebung selbständig bekannt und kann unmittelbar angesteuert werden.

Ziel des Projekts ist die Konzipierung und Entwicklung eines solchen Systems und die Erstellung einer Heizungssteuerung, die sich wie beschrieben einfach in die Hausautomatisierungsumgebung integrieren lässt. Dabei werden teilweise bereits existierende Konzepte der Vorgruppen übernommen.



