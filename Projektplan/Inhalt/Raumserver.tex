\subsection{Raumserver}\label{Raumserver}
Der Raumserver ist eine Java-Anwendung zur Steuerung von Geräten über ZigBee. Seine Befehle nimmt er über eine Message-Queue entgegen (siehe \autoref{Messaging}. Diese kann vom Benutzer über die Hausautomatisierungs-API vom Cloud Backend angesprochen werden. Der Raumserver bietet darüberhinaus eine Webseite zur Konfiguration (siehe \autoref{RoomConfig}). Zur Kommunikation mit ZigBee-fähigen Geräten wird ein generisches Protokoll verwendet (\autoref{Protokoll}). Der Raspberry Pi dient als Plattform für die Software. Am Raspberry Pi ist ein Temperatursensor verbaut, der vom Raumserver verwendet wird um die Raumtemperatur zu regulieren.

\subsubsection{Hardware}\label{Raum_Hardware}
In folgendem Abschnitt wird auf die hardwarespezifischen Besonderheiten des Raumservers eingegangen. Dabei wird in \autoref{Pi} auf die Plattform eingegangen auf der die Java-Anwendung ausgeführt wird. \autoref{Temp} beschreibt den Temperatursensor, der auf dem Pi verbaut ist. In \autoref{Raum_XBee} wird auf die Hardware eingegangen, die verwendet wird um über das ZigBee-Protokoll mit den Endgeräten zu kommunizieren. Abschließend wird in \autoref{Raum_LED} die Bedeutung der LED-Anzeigen erleutert.
\paragraph{Raspberry Pi}\mbox{}\\\label{Pi}

\paragraph{Temperatursensor}\mbox{}\\\label{Temp}

\paragraph{XBee USB Module}\mbox{}\\\label{Raum_XBee}

\paragraph{Status-LED}\mbox{}\\\label{Raum_LED}


\subsubsection{Software}\label{Raum_Software}

\subsubsection{Schnittstellen}\label{Raum_Schnittstellen}
Der Raumserver bietet mehrere Schnittstellen. Eine Übersicht ist in \Abbildung{RSSS} dargestellt. Dabei wird ein Heizungsventil beipielhaft als Endgerät angegeben.
Der Raumserver kommuniziert mit dem Cloud Backend über die bereitgestellte REST API (siehe \autoref{API}) mittels HTTPS. Über diesen Kanal werden vorwiegend Registrierungen von Endgeräten und Werteänderungen übertragen. Damit der Raumserver mit dem Cloud Backend kommunizieren kann muss er sich bei diesem authentifizieren. Dazu muss der jeweilige Cloud Account des Benutzers über die Konfigurationsseite des Raumservers eingetragen werden (siehe \autoref{RoomConfig}).

Die entgegengesetzte Kommunikation findet über das Messaging-Verfahren AMQP\footnote{https://www.amqp.org/} statt. Die Adressierung der Message Queues geschieht über die eindeutige ID der Raumserver. Die Queues sind persistent. Das heißt für den Fall, dass der Raumserver nicht erreichbar ist, bleiben die Nachrichten bestehen. Der Nachrichteninhalt ist im JSON-Fromat. Siehe dazu folgendes Beispiel:
\vspace{2em}
\begin{lstlisting}
{
   "deviceId": "123",
   "targetValue: "20.5"
}

\end{lstlisting}
\vspace{2em}
Nachrichten beschränken sich auf das Setzen von Werten für bestimmte Endgeräte. Das jeweilige Endgerät wird über die im JSON mitgelieferte deviceId identifiziert. Beim Raumserver eingehende Nachrichten werden an die entsprechenden Endgeräte über das ZigBee-Protokoll weitergeleitet (siehe dazu \autoref{ZigBee}). 

\begin{figure}[htb]
\centering
\includegraphicsKeepAspectRatio{Raumserver_Com.png}{0.6}
\caption{Schnittstellen des Raumservers}
\label{fig:RSSS}
\end{figure}

Innerhalb der ZigBee-Kommunikation nimmt der Raumserver die Rolle des Coordinator\footnote{https://en.wikipedia.org/wiki/ZigBee} ein und die Geräte wie zum Beispiel das Heizungsventil die Rolle des End Device (siehe \Abbildung{CBRSED}).

\begin{figure}[htb]
\centering
\includegraphicsKeepAspectRatio{rooms_graph2.png}{0.6}
\caption{ZigBee-Netzwerk mit Raumserver und Geräten}
\label{fig:CBRSED}
\end{figure}

\subsubsection{Konfiguration}\label{RoomConfig}

\begin{figure}[htb]
\centering
\includegraphicsKeepAspectRatio{Config.png}{1.0}
\caption{Konfigurationsseite des Raumservers}
\label{fig:ConfigRaum}
\end{figure}

\begin{figure}[htb]
\centering
\includegraphicsKeepAspectRatio{ConfigLogin.png}{1.0}
\caption{Eingabemaske für den Cloud Account}
\label{fig:ConfigRaum_Login}
\end{figure}






\subsubsection{Nachrichtenprotokoll}\label{Protokoll}

\begin{figure}[htb]
\centering
\includegraphicsKeepAspectRatio{protokoll.png}{0.6}
\caption{Raumserver XBee-Nachrichtenstruktur}
\label{fig:Frame}
\end{figure}


\begin{table}[]
\centering
\begin{tabular}{|l|l|l|l|}
\hline
\textbf{CMD} & \textbf{Bezeichnung} & \textbf{ARG1} & \textbf{ARG2} \\ \hline
01            & Wert setzen          & Wert          &               \\ \hline
02            & Statusanfrage        &               &               \\ \hline
03            & Status               & Gerätetype    & Wert          \\ \hline
\end{tabular}
\caption{Komanndotypen}
\label{CMDS}
\end{table}
\begin{table}[]
\centering
\begin{tabular}{|l|l|}
\hline
\textbf{Gerätetypen} & \textbf{Bezeichnung} \\ \hline
01                    & Heizung           \\ \hline
02                    & Schalter             \\ \hline
\end{tabular}
\caption{Gerätetypen}
\label{Devices}
\end{table}

\begin{table}[]
\centering
\begin{tabular}{|l|l|l|}
\hline
\textbf{CMD}                  & \textbf{ARG1}                 & \textbf{ARG2}                \\ \hline
\multicolumn{3}{|l|}{Statusanfrage}                                                          \\ \hline
02                            & 00                            & 00                           \\ \hline
\multicolumn{3}{|l|}{Statusantwort: Gerät ist Heizung mit Wert 0 =\textgreater geschlossen}  \\ \hline
03                            & 01                            & 00                           \\ \hline
\multicolumn{3}{|l|}{Heizung 5\% öffnen}                                                     \\ \hline
01                            & 05                            & 00                           \\ \hline
\multicolumn{3}{|l|}{Statusanfrage}                                                          \\ \hline
02                            & 00                            & 00                           \\ \hline
\multicolumn{3}{|l|}{Statusantwort: Gerät ist Heizung mit Wert 5 =\textgreater 5\% geöffnet} \\ \hline
03                            & 01                            & 05                           \\ \hline
\end{tabular}
\caption{Beispiel einer Kommunikation zwischen Raumserver und Heizung}
\label{PEx}
\end{table}

\subsubsection{Kommunikation über ZigBee}\label{ZigBee}

\subsubsection{Inbetriebnahme}

\paragraph{Inatallation}\mbox{}\\
alsdjflakjsdf

\paragraph{Starten und Stoppen}\mbox{}\\
lajd lajsdf laj d

\paragraph{Logging}\mbox{}\\
alsdjfla sd
