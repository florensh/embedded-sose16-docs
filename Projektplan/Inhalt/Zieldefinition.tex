\section{Zieldefinition}\label{Zieldefinition}
Gegenstand des Projekts ist die Weiterführung des über mehrere Semester bearbeitete Thema Hausautomatisierung mit ZigBee. Der Stand der Vorgruppe ist ein durchgeführter Proof-Of-Concept mit Entwicklung eines prototypischen Systems zur Steuerung eines Heizungsventils. Die Absicht der Weiterführung des Themas ist Verarbeitung der gewonnen Erkenntnisse und die Entwicklung eines einsetzbaren Hausautomatisierungssystems. Der Aspekt der Einsetzbarkeit ist dabei so definiert, dass das System Gütekriterien aktueller Produkte im Bereich der Hausautomatisierung entspricht. Als exemplarisches Endgerät, das über das System angesteuert werden kann, bleibt weiterhin das Heizungsventil im Fokus. Der Anspruch an dieses ist nun die Kapselung der Elektronik in ein geschlossenes Gehäuse, so dass ein Anschluss an einen Heizkörper möglich ist. Ein Überblick des Systems inform eines Use-Case-Diagramms ist in \Abbildung{UC} dargestellt. Im Folgendem werden die Teilziele bzw. die Gütekriterien des geplanten Systems erläutert.

\begin{figure}[htb]
\centering
\includegraphicsKeepAspectRatio{useCase.png}{0.9}
\caption{Use-Case-Diagramm Heizungssteuerung}
\label{fig:UC}
\end{figure}

\subsection{Erweiterbarkeit}\label{Ziel_Erw}
Die Erweiterbarkeit des Systems beinhaltet zum einen die Möglichkeit der Ansteuerung vielfältiger Endgeräte wie zum Beispiel Lichtschalter, Türöffner, Jalousien. Die Erweiterbarkeit in diesem Sinne erfordert generische Ansätze in den Bereichen der Kommunikation, Datenhaltung und Darstellung. Zum anderen wird Erweiterbarkeit verstanden als Möglichkeit das Softwaresystem ohne großen Aufwand um neue Anforderungen, wie zum Beispiel die Entwicklung von Benutzeroberflächen für verschiedene Endgeräte, zu ergänzen.
\subsection{Benutzerfreundlichkeit}\label{Ziel_Ben}
Die Bedienung, Installation und Erweiterung des Systems soll für den Benutzer möglichst einfach erlernbar sein. Die Benutzeroberflächen sollen dabei an Benutzer angepasst werden, die über keine internen Systemkenntnisse verfügen.
\subsection{Sicherheit}\label{Ziel_Sic}
Das System soll gegen unbefugten Zugriff von außen abgesichert sein. Komponenten der Hausautomatisierung dürfen nicht von Unberechtigten gesteuert werden können.

