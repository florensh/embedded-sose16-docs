\begin{tabularx}{\textwidth}{|l|X|}
\hline
Bezeichnung & Einen spezifischen Raum von der Datenbank entfernen\\ \hline
URL &  \colorbox{pregray}{\lstinline{/api/rooms/:id}}\\ \hline
Methode & DELETE \\ \hline
URL-Parameter & \textbf{erforderlich:}\newline \colorbox{pregray}{\lstinline{id=[alphanumeric]}} \newline Beispiel: id=d934eb20-4c6f-4d1c-91c5-61cdeddcf843 \\ \hline

Erfolgsantwort & 
\begin{lstlisting}^^J
HTTP/1.1 200 OK^^J
X-Content-Type-Options: nosniff^^J
X-XSS-Protection: 1; mode=block^^J
Cache-Control: no-cache, no-store, max-age=0, must-revalidate^^J
Pragma: no-cache^^J
Expires: 0^^J
X-Frame-Options: DENY^^J
Content-Type: text/plain;charset=UTF-8^^J
Content-Length: 25^^J
^^J
room successfully removed^^J
\end{lstlisting}\\ \hline
Fehlerantwort & 
\begin{lstlisting}^^J
HTTP/1.1 404 Not Found^^J
X-Content-Type-Options: nosniff^^J
X-XSS-Protection: 1; mode=block^^J
Cache-Control: no-cache, no-store, max-age=0, must-revalidate^^J
Pragma: no-cache^^J
Expires: 0^^J
X-Frame-Options: DENY^^J
Content-Type: text/plain;charset=UTF-8^^J
Content-Length: 14^^J
^^J
room not found^^J
\end{lstlisting}\\ \hline
Beispiel & 
\begin{lstlisting}^^J
$ curl 'http://localhost:8080/api/rooms/d934eb20-4c6f-4d1c-91c5-61cdeddcf843' -i -X DELETE -H 'X-AUTH-TOKEN: eyJpZCI6MTEsInVzZXJuYW1l...'^^J
\end{lstlisting}\\ \hline
\end{tabularx}

