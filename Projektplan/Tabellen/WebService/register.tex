\begin{tabularx}{\textwidth}{|l|X|}
\hline
Bezeichnung & Registrieren eines neuen Benutzers\\ \hline
URL &  \colorbox{pregray}{\lstinline{/api/register}}\\ \hline
Methode & POST \\ \hline


Parameter & 
\begin{lstlisting}^^J
\{^^J
  "username":[String],^^J
  "password":[alphanumeric]^^J
\}^^J
\end{lstlisting}\\ \hline


Erfolgsantwort & 
\begin{lstlisting}^^J
HTTP/1.1 200 OK^^J
X-Content-Type-Options: nosniff^^J
X-XSS-Protection: 1; mode=block^^J
Cache-Control: no-cache, no-store, max-age=0, must-revalidate^^J
Pragma: no-cache^^J
Expires: 0^^J
X-Frame-Options: DENY^^J
X-AUTH-TOKEN: eyJpZCI6MTIsInVz...^^J
Content-Type: application/json;charset=UTF-8^^J
Content-Length: 72^^J
^^J
user successfully registered^^J
\end{lstlisting}\\ \hline


Fehlerantwort & 
\begin{lstlisting}^^J
HTTP/1.1 422 Unprocessable Entity^^J
X-Content-Type-Options: nosniff^^J
X-XSS-Protection: 1; mode=block^^J
Cache-Control: no-cache, no-store, max-age=0, must-revalidate^^J
Pragma: no-cache^^J
Expires: 0^^J
X-Frame-Options: DENY^^J
Content-Type: text/plain;charset=UTF-8^^J
Content-Length: 17^^J
^^J
password to short^^J
\end{lstlisting}\\ \hline


Beispiel & 
\begin{lstlisting}^^J
$ curl 'http://localhost:8080/api/register' -i -X POST -H 'Content-Type: application/json' -d^^J '\{"username":"testuser", "password": "testpassword"\}'^^J
\end{lstlisting}\\ \hline
\end{tabularx}

